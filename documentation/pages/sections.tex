
\chapter{Sections}
\label{chap:sections}

Sections in FEMaster are used to define the material properties, geometry, and other characteristics of specific regions of a model. Sections are applied to element sets (ELSETs) to ensure that all elements within the set inherit the defined properties. The following section types are supported:

\begin{itemize}
\item \textbf{Solid Section (*SOLID SECTION)}
\item \textbf{Point Mass Section (*POINT MASS SECTION)}
\item \textbf{Shell Section (*SHELL SECTION)}
\item \textbf{Beam Section (*BEAM SECTION)}
\end{itemize}

Each section type has a unique set of parameters that can be specified in the input file. The following sections describe the syntax and parameters for each type.

% some spacing, not a new page

\section{Solid Section \texttt{(*SOLID SECTION)}}
Solid sections define the material properties for solid elements. These are primarily used for 3D solid elements like C3D4, C3D8, and C3D20.

\subsection{Syntax}
\begin{codeBlock}
*SOLID SECTION, ELSET=, MAT=
\end{codeBlock}

\begin{itemize}
\item \textbf{ELSET}: The name of the element set to which the solid section is applied. This is a required parameter.
\item \textbf{MAT} or \textbf{MATERIAL}: The name of the material to be used for the elements in the set. This is a required parameter.
\end{itemize}

\section{Point Mass Section \texttt{(*POINT MASS SECTION)}}
Point mass sections define mass and inertia properties for point elements. These sections are used to model concentrated masses or inertia effects.

\subsection{Syntax}
\begin{codeBlock}
*POINT MASS SECTION, ELSET=
<mass>
<inertia_x>, <inertia_y>, <inertia_z>
<spring_x>, <spring_y>, <spring_z>
<rotary_spring_x>, <rotary_spring_y>, <rotary_spring_z>
\end{codeBlock}

\begin{itemize}
\item \textbf{ELSET}: The name of the element set to which the point mass section is applied. This is a required parameter.
\item \textbf{Mass (optional)}: The concentrated mass. Defaults to 0.0 if not provided.
\item \textbf{Inertia (optional)}: The rotational inertia components in the x, y, and z directions. Defaults to .
\item \textbf{Spring (optional)}: Translational spring stiffness in the x, y, and z directions. Defaults to .
\item \textbf{Rotary Spring (optional)}: Rotational spring stiffness in the x, y, and z directions. Defaults to .
\end{itemize}

\section{Shell Section \texttt{(*SHELL SECTION)}}
Shell sections define the material properties and thickness for shell elements. These sections are used for elements like S4 and S8.

\subsection{Syntax}
\begin{codeBlock}
*SHELL SECTION, ELSET=, MAT=
<thickness>
\end{codeBlock}

\begin{itemize}
\item \textbf{ELSET}: The name of the element set to which the shell section is applied. This is a required parameter.
\item \textbf{MAT} or \textbf{MATERIAL}: The name of the material to be used for the elements in the set. This is a required parameter.
\item \textbf{thickness}: The thickness of the shell elements. This is a required parameter.
\end{itemize}

\section{Beam Section \texttt{(*BEAM SECTION)}}
Beam sections define the material properties and cross-sectional profile for beam elements. These sections are used for elements like B33.

\subsection{Syntax}
\begin{codeBlock}
*BEAM SECTION, ELSET=, MAT=, PROFILE=
<orientation_x>, <orientation_y>, <orientation_z>
\end{codeBlock}

\begin{itemize}
\item \textbf{ELSET}: The name of the element set to which the beam section is applied. This is a required parameter.
\item \textbf{MAT} or \textbf{MATERIAL}: The name of the material to be used for the elements in the set. This is a required parameter.
\item \textbf{PROFILE}: The name of the cross-sectional profile. This is a required parameter and must correspond to a profile defined using the \texttt{*PROFILE} command.
\item \textbf{Orientation (optional)}: A vector  defining the beam orientation.
\end{itemize}

\subsection{Defining Profiles}
Profiles define the geometric properties of beam cross-sections. They must be specified using the \texttt{*PROFILE} command.

\subsubsection{Syntax}
\begin{codeBlock}
*PROFILE, NAME=
<area>, <second_moment_of_area_y>, <second_moment_of_area_z>, <torsional_constant>
\end{codeBlock}

\begin{itemize}
\item \textbf{NAME} or \textbf{PROFILE}: The name of the profile. This is a required parameter.
\item \textbf{Area}: The cross-sectional area. Defaults to 0.0 if not provided.
\item \textbf{Moment of Inertia (Iy, Iz)}: The moments of inertia about the y and z axes. Defaults to 0.0 if not provided.
\item \textbf{Torsional Constant (Jt)}: The torsional constant. Defaults to 0.0 if not provided.
\end{itemize}




