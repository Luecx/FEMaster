\chapter{Boundary Conditions}
\label{chap:bc}

Boundary conditions (BCs) are essential in finite element analysis to define how a structure interacts with its environment and how loads are applied to it. This chapter explains the three main types of loads supported—`VLOAD`, `CLOAD`, and `DLOAD`—as well as the syntax for applying constraints using `*SUPPORT`. Boundary conditions and loads are grouped into *collectors* for easy management and reusability, enabling consistent application across multiple steps in an analysis.

\section{Load Types}

\subsection{Volume Load (*VLOAD)}

`VLOAD` is used to apply distributed loads to elements. This type of load is defined using three components: \( F_x, F_y, F_z \), which represent forces acting along the x, y, and z directions, respectively. `VLOAD` can be applied either to individual element IDs or to entire element sets. It is suitable for defining gravitational loads, body forces, or thermal forces distributed over the volume of elements.

\subsubsection{Syntax:}
\begin{codeBlock}
*VLOAD, LOAD_COLLECTOR=COLLECTOR_NAME
ELEMENT_ID, F_x, F_y, F_z
ELEMENT_SET_NAME, F_x, F_y, F_z
\end{codeBlock}

\textbf{Example:}
\begin{codeBlock}
*VLOAD, LOAD_COLLECTOR=GRAVITY
10, 0, -9.81, 0
11, 0, -9.81, 0
ELSET1, 0, -9.81, 0
\end{codeBlock}

\subsection{Concentrated Load (*CLOAD)}

`CLOAD` is used to apply concentrated loads to nodes. It supports six components: three translational forces \( F_x, F_y, F_z \) and three optional rotational moments \( M_x, M_y, M_z \).

\subsubsection{Syntax}

\begin{codeBlock}
*CLOAD, LOAD_COLLECTOR=COLLECTOR_NAME
NODE_ID, F_x, F_y, F_z, M_x, M_y, M_z
\end{codeBlock}

\textbf{Example:}

\begin{codeBlock}
*CLOAD, LOAD_COLLECTOR=LOADS
15, 0, 0, 1
16, 0, 0, 1
17, 0, 0, 1
18, 0, 0, 1
\end{codeBlock}

\subsection{Distributed Load (*DLOAD)}

`DLOAD` is used to apply distributed loads to surfaces. This load type is suitable for defining pressure loads or other distributed forces acting over surface areas.

\subsubsection{Syntax}

\begin{codeBlock}
*DLOAD, LOAD_COLLECTOR=COLLECTOR_NAME
SURFACE_ID, F_x, F_y, F_z
SURFACE_SET_NAME, F_x, F_y, F_z
\end{codeBlock}

\textbf{Example:}
\begin{codeBlock}
*DLOAD, LOAD_COLLECTOR=PRESSURE
20, 0, 0, -10
SURFSET1, 0, 0, -10
\end{codeBlock}

\section{Boundary Conditions (*SUPPORT)}

Boundary conditions specify which degrees of freedom (DOF) are fixed or constrained at a particular node. The `*SUPPORT` keyword is used to define constraints on any combination of the six available DOFs: three translational (\( u_x, u_y, u_z \)) and three rotational (\( \theta_x, \theta_y, \theta_z \)).

\subsection{Syntax}
\begin{codeBlock}
*SUPPORT, SUPPORT_COLLECTOR=COLLECTOR_NAME
NODE_ID, u_x, u_y, u_z, \theta_x, \theta_y, \theta_z
\end{codeBlock}

\textbf{Example:}
\begin{codeBlock}
*SUPPORT, SUPPORT_COLLECTOR=BCS
1, 0, 0, 0, 0, 0, 0
2, 0, 0, 0, 0, 0, 0
3, 0, 0, 0, 0, 0, 0
4, 0, 0, 0, 0, 0, 0
\end{codeBlock}

\section{Load Case Management}

Once loads and supports are grouped into collectors, they can be easily referenced in different load cases within a step.
Load cases define how the loads and supports are applied, enabling different scenarios.

\subsection{Syntax}
\begin{codeBlock}
*LOADCASE, TYPE=LINEAR STATIC
*LOAD
LOAD_COLLECTOR_1
LOAD_COLLECTOR_2
*SUPPORT
SUPPORT_COLLECTOR_1
SUPPORT_COLLECTOR_2
\end{codeBlock}

\section{Conclusion}

Boundary conditions and loads are fundamental to setting up a finite element analysis.
The use of `VLOAD`, `CLOAD`, and `DLOAD` allows for a wide range of loading scenarios,
while the `*SUPPORT` keyword enables precise control over node constraints.
By grouping these into collectors, users can efficiently manage and apply conditions across multiple steps,
enhancing flexibility and reusability in complex analyses.