
\chapter{Mathematical Derivation}
\label{chap:derivation}

The stiffness matrix of an element in FEM is defined as:

\begin{equation}
\label{eq:stiffness_matrix}
\mathbf{K}_{e,0} = \int_{\Omega} \mathbf{B}^T \mathbf{D} \mathbf{B} \, d\Omega
\end{equation}

where \(\mathbf{K}_{e,0}\) is the stiffness matrix of an element, \(\mathbf{B}\) is the strain-displacement matrix, \(\mathbf{D}\) is the constitutive matrix, and \(\Omega\) is the volume of the element.

Using the principle of isoparametric elements, the integration becomes:

\begin{equation}
\label{eq:stiffness_matrix2}
\mathbf{K}_e = \int_{\Omega'} \mathbf{B}^T \mathbf{D} \mathbf{B} \, |\text{det}(\mathbf{J})| \, d\Xi \, d\eta \, d\zeta
\end{equation}

where \(\Omega'\) is the volume of the reference element, \(\mathbf{J}\) is the Jacobian matrix, and \(\Xi\), \(\eta\), and \(\zeta\) are the natural coordinates of the reference element. For a simple 2D quadrilateral element, the domain \(\Omega'\) is commonly defined as: \(-1 \leq \Xi \leq 1\) and \(-1 \leq \eta \leq 1\).

From the principle of virtual work, the external work done is defined as:

\begin{equation}
\label{eq:compliance}
\mathbf{W} = \mathbf{F} \cdot \mathbf{u}
\end{equation}

where \(\mathbf{W}\) is the external work done, \(\mathbf{F}\) is the external force vector, and \(\mathbf{u}\) is the displacement vector.

The vector \(\mathbf{F}\) can be expressed in terms of the global stiffness matrix and the displacement vector of the entire structure as:

\begin{equation}
\label{eq:force}
\mathbf{F} = \mathbf{K} \mathbf{u}
\end{equation}

where \(\mathbf{K}\) is the global stiffness matrix of the structure.

Using this, the external work done can be expressed as:

\begin{align}
\label{eq:compliance2}
\mathbf{W} &= \mathbf{F} \cdot \mathbf{u} \\
           &= \mathbf{u}^T \mathbf{K} \mathbf{u}
\end{align}

Since the internal work must equal the external work, the internal work, called the compliance \(\mathbf{C}\), can be expressed as:

\begin{equation}
\label{eq:compliance3}
\mathbf{C} = \mathbf{u}^T \mathbf{K} \mathbf{u}
\end{equation}

The compliance can be expressed as a function of the stiffness matrices of each element. Based on the assembly of the global matrix \(\mathbf{K}\), the compliance can be expressed as:

\begin{equation}
\mathbf{C} = \sum_{e=1}^{n} \mathbf{u}_e^T \mathbf{K}_e \mathbf{u}_e
\end{equation}

where \(\mathbf{u}_e\) is the displacement vector of element \(e\) and \(\mathbf{K}_e\) is the stiffness matrix of element \(e\).
This shows that \textit{the compliance, which is the strain energy of the object, is the sum of the strain energies of each element}.
This forms the basis for the topology optimization problem.

\section{Derivative with Respect to the Density}

In topology optimization, compliance is often minimized subject to volume constraints. The design variables are the densities of each element. A virtual density, \(\rho\), is used to adjust the stiffness matrix of each element. Together with an exponent \(p\), the equation for the stiffness matrix of an element becomes:

\begin{align}
\label{eq:stiffness_matrix3}
\mathbf{K}_e &= \rho^p \int_{\Omega'} \mathbf{B}^T \mathbf{D} \mathbf{B} \, |\text{det}(\mathbf{J})| \, d\Xi \, d\eta \, d\zeta \\
             &= \rho^p \mathbf{K}_{e,0}
\end{align}

The parameter \(p\) is used to encourage the optimizer toward a binary solution and will not be further discussed in this document.

Assuming that \(\mathbf{K}_{e,0}\) is constant, the derivative of the compliance with respect to the density can be expressed as:

\begin{equation}
\label{eq:compliance_derivative}
\frac{\partial \mathbf{C}}{\partial \rho_i} = p \rho_i^{p - 1} \cdot \left( \mathbf{u}_e^T \mathbf{K}_{e,0} \mathbf{u}_e \right)
\end{equation}

which is straightforward to compute. With advancements in manufacturing, the material orientation can vary within the geometry.
Incorporating this in the topology optimization problem requires formulating the derivative of compliance with respect to the material orientation.

\section{Derivative with Respect to the Material Orientation}

Deriving the compliance with respect to the material orientation requires the chain rule of differentiation.
In this document, it is assumed that the orientation of each element can be expressed using three angles \(\alpha_{1}\), \(\alpha_{2}\), and \(\alpha_{3}\)
that define the rotation of the material about the \(x\), \(y\), and \(z\) axes, respectively.

For this, the rotation matrices \(\mathbf{R}_1\), \(\mathbf{R}_2\), and \(\mathbf{R}_3\) are defined as:

\begin{align}
\mathbf{R}_1 &= \begin{bmatrix}
1 & 0 & 0 \\
0 & \cos(\alpha_{1}) & -\sin(\alpha_{1}) \\
0 & \sin(\alpha_{1}) & \cos(\alpha_{1})
\end{bmatrix} \\
\mathbf{R}_2 &= \begin{bmatrix}
\cos(\alpha_{2}) & 0 & \sin(\alpha_{2}) \\
0 & 1 & 0 \\
-\sin(\alpha_{2}) & 0 & \cos(\alpha_{2})
\end{bmatrix} \\
\mathbf{R}_3 &= \begin{bmatrix}
\cos(\alpha_{3}) & -\sin(\alpha_{3}) & 0 \\
\sin(\alpha_{3}) & \cos(\alpha_{3}) & 0 \\
0 & 0 & 1
\end{bmatrix}
\end{align}

The rotation matrices around all three axes are then combined to form the rotation matrix \(\mathbf{R}\) as:

\begin{equation}
\mathbf{R} = \mathbf{R}_1 \mathbf{R}_2 \mathbf{R}_3
\end{equation}

It is noted that the order of multiplication is irrelevant, both for the derivation and the final result, yet shall be consistent throughout the document.
Rotating the material matrix is done by transforming the strains and stresses. In Voigt notation, the strain and stress vectors are defined as:

\begin{align}
\mathbf{\varepsilon} &= \begin{bmatrix}
\varepsilon_{xx} \\
\varepsilon_{yy} \\
\varepsilon_{zz} \\
2\varepsilon_{yz} \\
2\varepsilon_{xz} \\
2\varepsilon_{xy}
\end{bmatrix} \\
\mathbf{\sigma} &= \begin{bmatrix}
\sigma_{xx} \\
\sigma_{yy} \\
\sigma_{zz} \\
\sigma_{yz} \\
\sigma_{xz} \\
\sigma_{xy}
\end{bmatrix}
\end{align}

The rotation of the strain vector to the material orientation is done using the transformation matrix \(\mathbf{T}\) which
can be expressed using the nine components of the rotation matrix \(\mathbf{R}\) as:

\begin{equation}
\mathbf{T} = \begin{bmatrix}
R_{11}^2 & R_{21}^2 & R_{31}^2 & R_{11} R_{21} & R_{21} R_{31} & R_{31} R_{11} \\
R_{12}^2 & R_{22}^2 & R_{32}^2 & R_{12} R_{22} & R_{22} R_{32} & R_{32} R_{12} \\
R_{13}^2 & R_{23}^2 & R_{33}^2 & R_{13} R_{23} & R_{23} R_{33} & R_{33} R_{13} \\
2 R_{11} R_{12} & 2 R_{21} R_{22} & 2 R_{31} R_{32} & R_{11} R_{22} + R_{12} R_{21} & R_{21} R_{32} + R_{31} R_{22} & R_{31} R_{12} + R_{32} R_{11} \\
2 R_{12} R_{13} & 2 R_{22} R_{23} & 2 R_{32} R_{33} & R_{23} R_{12} + R_{13} R_{22} & R_{22} R_{33} + R_{32} R_{23} & R_{32} R_{13} + R_{12} R_{33} \\
2 R_{11} R_{13} & 2 R_{21} R_{23} & 2 R_{31} R_{33} & R_{13} R_{21} + R_{11} R_{23} & R_{23} R_{31} + R_{21} R_{33} & R_{33} R_{11} + R_{31} R_{13}
\end{bmatrix}
\end{equation}

Finally, the rotation of the material matrix is done using the transformation matrix \(\mathbf{T}\) as:
\begin{equation}
\mathbf{D_{\text{rot}}} = \mathbf{T}^T \mathbf{D} \mathbf{T}
\end{equation}

Putting this all together yields the stiffness matrix for an element as:

\begin{equation}
\mathbf{K}_e = \rho^p \cdot \int_{\Omega'} \mathbf{B}^T \mathbf{D_{\text{rot}}} \mathbf{B} \, |\text{det}(\mathbf{J})| \, d\Xi \, d\eta \, d\zeta
\end{equation}

Taking the deriative of the compliance with respect to the material orientation is not straight forward because of the integral.
Since the integral will be replaced with numerical approximations, the above stiffness matrix can be reformulated as.
Using gauss integration with \(Q\) integration points, and the assumption that both, $\mathbf{B}$ and $\mathbf{J}$ are a function of the local coordinates $\mathbf{r} = (\Xi, \eta, \zeta)$, the stiffness matrix can be expressed as:

\begin{equation}
\mathbf{K}_e = \rho^p \cdot \sum_{q=1}^Q w_q \mathbf{B(\mathbf{r}_q)}^T \mathbf{D_{\text{rot}}} \mathbf{B(\mathbf{r}_q)} \cdot \text{det}(\mathbf{J(\mathbf{r}_q)})
\end{equation}

where $w_q$ are the weights of the gauss integration and $\mathbf{r}_q$ are the local coordinates of the gauss points.
Plugging this back into the equation for the compliance for a single element $i$ yields:

\begin{align}
C_i = \mathbf{u_i}^T \mathbf{K}_i \mathbf{u_i} = \mathbf{u_i}^T \left( \rho_i^p \sum_{q=1}^Q w_q \mathbf{B(\mathbf{r}_q)}^T \mathbf{D_{\text{rot}}} \mathbf{B(\mathbf{r}_q)} \cdot \text{det}(\mathbf{J(\mathbf{r}_q)}) \right) \mathbf{u_i} \\
    = \mathbf{u_i}^T \left( \rho_i^p \sum_{q=1}^Q w_q \mathbf{B(\mathbf{r}_q)}^T \mathbf{T}^T \mathbf{D} \mathbf{T} \mathbf{B(\mathbf{r}_q)} \cdot \text{det}(\mathbf{J(\mathbf{r}_q)}) \right) \mathbf{u_i}\\
    = \rho_i^p \sum_{q=1}^Q w_q \mathbf{u_i}^T \mathbf{B(\mathbf{r}_q)}^T \mathbf{T}^T \mathbf{D} \mathbf{T} \mathbf{B(\mathbf{r}_q)} \cdot \text{det}(\mathbf{J(\mathbf{r}_q)}) \mathbf{u_i}\\
    = \rho_i^p \sum_{q=1}^Q w_q (\mathbf{B(\mathbf{r}_q)} \mathbf{u_i})^T  \mathbf{T}^T \mathbf{D} \mathbf{T} (\mathbf{B(\mathbf{r}_q)} \mathbf{u_i} ) \cdot \text{det}(\mathbf{J(\mathbf{r}_q)})\\
    = \rho_i^p \sum_{q=1}^Q w_q \mathbf{\varepsilon(\mathbf{r}_q)}^T  \mathbf{T}^T \mathbf{D} \mathbf{T} \mathbf{\varepsilon(\mathbf{r}_q)} \cdot \text{det}(\mathbf{J(\mathbf{r}_q)})\\
\end{align}

Since only $T$ depends on the material orientation, the derivative of the compliance with respect to the material orientation can be expressed as:

\begin{align}
\frac{\partial C_i}{\partial \alpha_{1}} = \rho_i^p \sum_{q=1}^Q w_q \frac{\partial}{\partial \alpha_{1}} \left( \mathbf{\varepsilon(\mathbf{r}_q)}^T  \mathbf{T}^T \mathbf{D} \mathbf{T} \mathbf{\varepsilon(\mathbf{r}_q)} \cdot \text{det}(\mathbf{J(\mathbf{r}_q)}) \right)\\
\end{align}

The only thing left is the expression of the derivative of $T$ with respect to the material orientation.
The matrix \( T \) derivative with respect to \( \alpha_{j} \) becomes:

\begin{align}
\frac{\partial \mathbf{T}}{\partial \alpha_{j}} =
\begin{bmatrix}
2 R_{11} R_{11,j} & 2 R_{21} R_{21,j} & 2 R_{31} R_{31,j} \\ R_{11} R_{21,j} + R_{11,j} R_{21} & R_{21} R_{31,j} + R_{21,j} R_{31} & R_{31} R_{11,j} + R_{31,j} R_{11} \\
\\
2 R_{12} R_{12,j} & 2 R_{22} R_{22,j} & 2 R_{32} R_{32,j} \\ R_{12} R_{22,j} + R_{12,j} R_{22} & R_{22} R_{32,j} + R_{22,j} R_{32} & R_{32} R_{12,j} + R_{32,j} R_{12} \\
\\
2 R_{13} R_{13,j} & 2 R_{23} R_{23,j} & 2 R_{33} R_{33,j} \\ R_{13} R_{23,j} + R_{13,j} R_{23} & R_{23} R_{33,j} + R_{23,j} R_{33} & R_{33} R_{13,j} + R_{33,j} R_{13} \\
\\
2 (R_{11} R_{12,j} + R_{11,j} R_{12}) &
    2 (R_{21} R_{22,j} + R_{21,j} R_{22}) &
    2 (R_{31} R_{32,j} + R_{31,j} R_{32}) \\
    R_{11,j} R_{22} + R_{11} R_{22,j} + R_{12,j} R_{21} + R_{12} R_{21,j} &
    R_{21,j} R_{32} + R_{21} R_{32,j} + R_{22,j} R_{31} + R_{22} R_{31,j} &
    R_{31,j} R_{12} + R_{31} R_{12,j} + R_{32,j} R_{11} + R_{32} R_{11,j} \\
    \\
2 (R_{12} R_{13,j} + R_{12,j} R_{13}) &
    2 (R_{22} R_{23,j} + R_{22,j} R_{23}) &
    2 (R_{32} R_{33,j} + R_{32,j} R_{33}) \\
    R_{23,j} R_{12} + R_{23} R_{12,j} + R_{13,j} R_{22} + R_{13} R_{22,j} &
    R_{22,j} R_{33} + R_{22} R_{33,j} + R_{32,j} R_{23} + R_{32} R_{23,j} &
    R_{32,j} R_{13} + R_{32} R_{13,j} + R_{33,j} R_{12} + R_{33} R_{12,j} \\
    \\
2 (R_{11} R_{13,j} + R_{11,j} R_{13}) &
    2 (R_{23} R_{21,j} + R_{23,j} R_{21}) &
    2 (R_{33} R_{31,j} + R_{33,j} R_{31}) \\
    R_{13,j} R_{21} + R_{13} R_{21,j} + R_{11,j} R_{23} + R_{11} R_{23,j} &
    R_{23,j} R_{31} + R_{23} R_{31,j} + R_{21,j} R_{33} + R_{21} R_{33,j} &
    R_{33,j} R_{11} + R_{33} R_{11,j} + R_{31,j} R_{13} + R_{31} R_{13,j}\\
\end{bmatrix}
\end{align}


The rotation matrices \( \mathbf{R}_1 \), \( \mathbf{R}_2 \), and \( \mathbf{R}_3 \), each dependent only on one respective angle \( \alpha_{j} \), now have their derivatives as follows:

\begin{equation}
\mathbf{R}_{1,j} = \frac{\partial \mathbf{R}_1}{\partial \alpha_{1}} = \begin{bmatrix}
0 & 0 & 0 \\
0 & -\sin(\alpha_{1}) & -\cos(\alpha_{1}) \\
0 & \cos(\alpha_{1}) & -\sin(\alpha_{1})
\end{bmatrix}
\end{equation}

\begin{equation}
\mathbf{R}_{2,j} = \frac{\partial \mathbf{R}_2}{\partial \alpha_{2}} = \begin{bmatrix}
-\sin(\alpha_{2}) & 0 & \cos(\alpha_{2}) \\
0 & 0 & 0 \\
-\cos(\alpha_{2}) & 0 & -\sin(\alpha_{2})
\end{bmatrix}
\end{equation}

\begin{equation}
\mathbf{R}_{3,j} = \frac{\partial \mathbf{R}_3}{\partial \alpha_{3}} = \begin{bmatrix}
-\sin(\alpha_{3}) & -\cos(\alpha_{3}) & 0 \\
\cos(\alpha_{3}) & -\sin(\alpha_{3}) & 0 \\
0 & 0 & 0
\end{bmatrix}
\end{equation}

Thus, the full rotation matrix \( \mathbf{R} \) with respect to each angle \( \alpha_{j} \) becomes:
\begin{align}
\frac{\partial \mathbf{R}}{\partial \alpha_{1}} &= \mathbf{R}_{1,j} \mathbf{R}_2 \mathbf{R}_3 \\
\frac{\partial \mathbf{R}}{\partial \alpha_{2}} &= \mathbf{R}_1 \mathbf{R}_{2,j} \mathbf{R}_3 \\
\frac{\partial \mathbf{R}}{\partial \alpha_{3}} &= \mathbf{R}_1 \mathbf{R}_2 \mathbf{R}_{3,j}
\end{align}


